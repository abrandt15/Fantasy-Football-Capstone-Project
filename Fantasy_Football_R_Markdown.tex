\documentclass[]{article}
\usepackage{lmodern}
\usepackage{amssymb,amsmath}
\usepackage{ifxetex,ifluatex}
\usepackage{fixltx2e} % provides \textsubscript
\ifnum 0\ifxetex 1\fi\ifluatex 1\fi=0 % if pdftex
  \usepackage[T1]{fontenc}
  \usepackage[utf8]{inputenc}
\else % if luatex or xelatex
  \ifxetex
    \usepackage{mathspec}
  \else
    \usepackage{fontspec}
  \fi
  \defaultfontfeatures{Ligatures=TeX,Scale=MatchLowercase}
\fi
% use upquote if available, for straight quotes in verbatim environments
\IfFileExists{upquote.sty}{\usepackage{upquote}}{}
% use microtype if available
\IfFileExists{microtype.sty}{%
\usepackage{microtype}
\UseMicrotypeSet[protrusion]{basicmath} % disable protrusion for tt fonts
}{}
\usepackage[margin=1in]{geometry}
\usepackage{hyperref}
\hypersetup{unicode=true,
            pdftitle={Fantasy Football Capstone Project},
            pdfauthor={Alex Brandt},
            pdfborder={0 0 0},
            breaklinks=true}
\urlstyle{same}  % don't use monospace font for urls
\usepackage{color}
\usepackage{fancyvrb}
\newcommand{\VerbBar}{|}
\newcommand{\VERB}{\Verb[commandchars=\\\{\}]}
\DefineVerbatimEnvironment{Highlighting}{Verbatim}{commandchars=\\\{\}}
% Add ',fontsize=\small' for more characters per line
\usepackage{framed}
\definecolor{shadecolor}{RGB}{248,248,248}
\newenvironment{Shaded}{\begin{snugshade}}{\end{snugshade}}
\newcommand{\KeywordTok}[1]{\textcolor[rgb]{0.13,0.29,0.53}{\textbf{#1}}}
\newcommand{\DataTypeTok}[1]{\textcolor[rgb]{0.13,0.29,0.53}{#1}}
\newcommand{\DecValTok}[1]{\textcolor[rgb]{0.00,0.00,0.81}{#1}}
\newcommand{\BaseNTok}[1]{\textcolor[rgb]{0.00,0.00,0.81}{#1}}
\newcommand{\FloatTok}[1]{\textcolor[rgb]{0.00,0.00,0.81}{#1}}
\newcommand{\ConstantTok}[1]{\textcolor[rgb]{0.00,0.00,0.00}{#1}}
\newcommand{\CharTok}[1]{\textcolor[rgb]{0.31,0.60,0.02}{#1}}
\newcommand{\SpecialCharTok}[1]{\textcolor[rgb]{0.00,0.00,0.00}{#1}}
\newcommand{\StringTok}[1]{\textcolor[rgb]{0.31,0.60,0.02}{#1}}
\newcommand{\VerbatimStringTok}[1]{\textcolor[rgb]{0.31,0.60,0.02}{#1}}
\newcommand{\SpecialStringTok}[1]{\textcolor[rgb]{0.31,0.60,0.02}{#1}}
\newcommand{\ImportTok}[1]{#1}
\newcommand{\CommentTok}[1]{\textcolor[rgb]{0.56,0.35,0.01}{\textit{#1}}}
\newcommand{\DocumentationTok}[1]{\textcolor[rgb]{0.56,0.35,0.01}{\textbf{\textit{#1}}}}
\newcommand{\AnnotationTok}[1]{\textcolor[rgb]{0.56,0.35,0.01}{\textbf{\textit{#1}}}}
\newcommand{\CommentVarTok}[1]{\textcolor[rgb]{0.56,0.35,0.01}{\textbf{\textit{#1}}}}
\newcommand{\OtherTok}[1]{\textcolor[rgb]{0.56,0.35,0.01}{#1}}
\newcommand{\FunctionTok}[1]{\textcolor[rgb]{0.00,0.00,0.00}{#1}}
\newcommand{\VariableTok}[1]{\textcolor[rgb]{0.00,0.00,0.00}{#1}}
\newcommand{\ControlFlowTok}[1]{\textcolor[rgb]{0.13,0.29,0.53}{\textbf{#1}}}
\newcommand{\OperatorTok}[1]{\textcolor[rgb]{0.81,0.36,0.00}{\textbf{#1}}}
\newcommand{\BuiltInTok}[1]{#1}
\newcommand{\ExtensionTok}[1]{#1}
\newcommand{\PreprocessorTok}[1]{\textcolor[rgb]{0.56,0.35,0.01}{\textit{#1}}}
\newcommand{\AttributeTok}[1]{\textcolor[rgb]{0.77,0.63,0.00}{#1}}
\newcommand{\RegionMarkerTok}[1]{#1}
\newcommand{\InformationTok}[1]{\textcolor[rgb]{0.56,0.35,0.01}{\textbf{\textit{#1}}}}
\newcommand{\WarningTok}[1]{\textcolor[rgb]{0.56,0.35,0.01}{\textbf{\textit{#1}}}}
\newcommand{\AlertTok}[1]{\textcolor[rgb]{0.94,0.16,0.16}{#1}}
\newcommand{\ErrorTok}[1]{\textcolor[rgb]{0.64,0.00,0.00}{\textbf{#1}}}
\newcommand{\NormalTok}[1]{#1}
\usepackage{graphicx,grffile}
\makeatletter
\def\maxwidth{\ifdim\Gin@nat@width>\linewidth\linewidth\else\Gin@nat@width\fi}
\def\maxheight{\ifdim\Gin@nat@height>\textheight\textheight\else\Gin@nat@height\fi}
\makeatother
% Scale images if necessary, so that they will not overflow the page
% margins by default, and it is still possible to overwrite the defaults
% using explicit options in \includegraphics[width, height, ...]{}
\setkeys{Gin}{width=\maxwidth,height=\maxheight,keepaspectratio}
\IfFileExists{parskip.sty}{%
\usepackage{parskip}
}{% else
\setlength{\parindent}{0pt}
\setlength{\parskip}{6pt plus 2pt minus 1pt}
}
\setlength{\emergencystretch}{3em}  % prevent overfull lines
\providecommand{\tightlist}{%
  \setlength{\itemsep}{0pt}\setlength{\parskip}{0pt}}
\setcounter{secnumdepth}{0}
% Redefines (sub)paragraphs to behave more like sections
\ifx\paragraph\undefined\else
\let\oldparagraph\paragraph
\renewcommand{\paragraph}[1]{\oldparagraph{#1}\mbox{}}
\fi
\ifx\subparagraph\undefined\else
\let\oldsubparagraph\subparagraph
\renewcommand{\subparagraph}[1]{\oldsubparagraph{#1}\mbox{}}
\fi

%%% Use protect on footnotes to avoid problems with footnotes in titles
\let\rmarkdownfootnote\footnote%
\def\footnote{\protect\rmarkdownfootnote}

%%% Change title format to be more compact
\usepackage{titling}

% Create subtitle command for use in maketitle
\newcommand{\subtitle}[1]{
  \posttitle{
    \begin{center}\large#1\end{center}
    }
}

\setlength{\droptitle}{-2em}

  \title{Fantasy Football Capstone Project}
    \pretitle{\vspace{\droptitle}\centering\huge}
  \posttitle{\par}
    \author{Alex Brandt}
    \preauthor{\centering\large\emph}
  \postauthor{\par}
      \predate{\centering\large\emph}
  \postdate{\par}
    \date{December 30, 2018}


\begin{document}
\maketitle

\subsection{Introduction}\label{introduction}

Fantasy football is a competitive game that is played by millions of
friends and colleagues every football season. The goal is to score more
points that your opponent each week by putting together the best
possible team of quarterbacks, running backs, wide receivers, tight
ends, kickers, and defense. While there are several different formats, I
will be using data from the PPR (Points per reception) format.

\subsection{Problem Statement}\label{problem-statement}

There are many different factors that can impact your final standing in
your league, however, I will be focusing on starting the season off
strong by having the best possible draft. I will be using the final 2017
data from Fantasydata.com to try and predict which players have the best
value. This data can be used before the season by any potential fantasy
league player trying to determine when to draft specific players in
order to score the most points in any given week during the season.
While adding undervalued players through the waiver wire during the
season is an important part to any winning team, this project is only
focusing on the initial draft. Will drafting Todd Gurley with the first
overall pick help my team make the playoffs? Should I draft Christian
Mccaffrey in the first round, or wait until my second pick? These are
some of the questions this project will attempt to answer.

I will be narrowing the scope of this project to focus specifically on
the running back position. Using the ADP (Average Draft Position) for
the 2018 season, I will be able to test the model using the previous
year's data.

\subsection{Data Wrangling}\label{data-wrangling}

The structure of the dataset was relatively clean, however, I removed
any players that had zero in the total fantasy points column. These are
players that didn't play enough to generate any material statistics
during the season. These outliers are not helpful in trying to predict
draft picks for the next season since they didn't earn any points in the
2017 season. Also, any player who scored negative fantasy points for the
entire season, which could be an indication of incorrect data, was
removed from the dataset.

I also removed a few columns from the dataset that were not relevant to
predicting which player would score the most points. Since the entire
dataset consists of only running backs, I removed the position column
since this was redundant. I also removed the fumbles column, since only
the fumbles lost column is needed when determining the amount points
scored. A player loses two points for each fumble lost. A fumble that is
recovered by the same team does not cause the player to lose any points.

Finally, I joined the average draft position dataset with the main
running back dataset, so that there was only one final dataset to work
with when diving into plotting and modeling.

\subsection{Data Visualization}\label{data-visualization}

With the data cleaned up, the next steps involved taking a deeper dive
into the data. I decided to create a few different plots to see if
anything jumped out right away. The first relationship I looked at was
rushing attempts and rushing yards.

\includegraphics{Fantasy_Football_R_Markdown_files/figure-latex/Rushing-1.pdf}

As you can see in the plot, there appears to be a linear relationship
between the number of rushing attempts a player has over the course of
the season and the total number of yards that player rushes for. We will
test this theory later, but since more yards equals more points, it
initially seems that we would want to focus on players that get the ball
more over the course of the season. The team each player is on can have
a huge impact on this statistic because it depends on offense strategy
that the team implements whether or not they are going to run the ball.
As you can see from the plot, in 2017 it seems like Kareem Hunt (on the
Kansas City Chiefs), Le'Veon Bell (on the Pittsburgh Steelers), Todd
Gurley (on the Los Angeles Rams), and Jordan Howard (on the Chicago
Bears) are among the top players with the most rushing attempts.

Another important relationship to consider when trying to predict total
fantasy points scored is between receiving targets and receiving yards.
In PPR (Points Per Reception), this is even more important since the
player receives an extra point for every reception they make. From the
graph below, the first thing that jumped out was that this relationship
was similar to the rushing attempts and rushing yards relationship in
that the more receiving targets a player gets throughout the season the
more yards they typically gain. They both have a positive linear
relationship.

\includegraphics{Fantasy_Football_R_Markdown_files/figure-latex/unnamed-chunk-1-1.pdf}

You can also see that there is a strong correlation between both these
relationships, however, there is a slightly stronger one between rushing
attempts and rushing yards. This makes sense because typically the more
opportunities a player gets to gain yards (both rushing and receiving),
the higher chance there is that he will score more points at the end of
the season.

\begin{Shaded}
\begin{Highlighting}[]
\KeywordTok{cor}\NormalTok{(rb}\OperatorTok{$}\NormalTok{RushingAttempts, rb}\OperatorTok{$}\NormalTok{RushingYards, }\DataTypeTok{method=}\StringTok{"spearman"}\NormalTok{)}
\end{Highlighting}
\end{Shaded}

\begin{verbatim}
## [1] 0.9872207
\end{verbatim}

\begin{Shaded}
\begin{Highlighting}[]
\KeywordTok{cor}\NormalTok{(rb}\OperatorTok{$}\NormalTok{ReceivingTargets, rb}\OperatorTok{$}\NormalTok{ReceivingYards, }\DataTypeTok{method=}\StringTok{"spearman"}\NormalTok{)}
\end{Highlighting}
\end{Shaded}

\begin{verbatim}
## [1] 0.9554589
\end{verbatim}

\subsection{Average Draft Position}\label{average-draft-position}

After deciding that these four statstics (rushing yards, receiving
yards, rushing attempts, receiving attempts) were the most important in
determining how many fantasy points a player will score by the end of
the season, I then plotted each of these with the average draft position
the following year to see which one of these was the best predictor of
future draft position.

\includegraphics{Fantasy_Football_R_Markdown_files/figure-latex/Rushing_Yards_and_Average_Draft_Position-1.pdf}

From the rushing yards plot, it appears the majority of the highest
total rushing yard players (above the line) were all drafted within the
top 25 spots. There were a few outliers, however, which could be due to
a couple different scenarios. First of all, if a star player was injured
for all of the previous season and is now healthy, then he could still
be drafted in the top ten this year, even if he didn't gain any rushing
yards the previous season. Also, a highly rated rookie that is expected
to perform well over the course of the upcoming season could still be
drafted in the first or second round of the fantasy draft. There is a
risk in implementing this strategy, however, since there are no pro
level statistics available to analyse. A league member who does draft a
rookie is expecting them to perform as well at the pro level as they did
on the collegiate level. A great example of this scenario in the current
year is Saquon Barkley. He was one of the highest drafted running backs
this year even though he was a rookie.

\includegraphics{Fantasy_Football_R_Markdown_files/figure-latex/receiving_yards-1.pdf}

The receiving yards plot turned out to be a little more scattered. While
the players with the two highest receiving yards were still drafted in
the top ten, the rest of the chart was little more difficult to predict.
There were several players with less receiving yards than average that
were still drafted high. It's important to remember that the scope of
this project is only looking at the running back position, so it make
sense that total rushing yards would be a little more correlated with
draft position. While there are still several running backs who catch a
lot out of the backfield, this position is more focused on running the
ball.

\includegraphics{Fantasy_Football_R_Markdown_files/figure-latex/rushing_attempts-1.pdf}

The next plot I reviewed was the rushing attempts by a player and the
average draft position. This represented the total opportunites a player
had to run the ball throughout the season. While the total rushing
attempts are still important to review, it appears this category is not
as statically important as the total rushing yards. Even though there is
a strong positive linear relationship between rushing attempts and
rushing yards, when it comes to actually predicting the average draft
postion, the total rushing yards is more important.

\includegraphics{Fantasy_Football_R_Markdown_files/figure-latex/receiving_targets-1.pdf}

The last plot I decided to look at was the receiving targets. This is
the amount of times the running back was targeted in passing game. The
receiving targets are similar to the rushing attempts in that each of
these plots are more scattered and not as materially important as the
total yards a player gains in determing the average draft position the
following year.

\subsection{Linear Models}\label{linear-models}

In order to determine which of these categories was the most staticially
significant, I ran several diffent linear models to see which of these
was the best predictor for the average draft position. The first
predictor I looked at was rushing yards. In this model, the rushing
yards are significant indictor of average draft position due to the low
std. error and very low probablity that the null hypothesis would be
true. In other words, there is very little chance of this result
happening just due to random variation and should be considered
statitically significant. This model also had a multiple R-squared of
.512, which will be need to be compared to the other models.

\begin{verbatim}
## 
## Call:
## lm(formula = ADP ~ RushingYards, data = rb)
## 
## Residuals:
##     Min      1Q  Median      3Q     Max 
## -86.912 -17.216   1.804  17.747  56.139 
## 
## Coefficients:
##               Estimate Std. Error t value Pr(>|t|)    
## (Intercept)  91.143714   4.040546   22.56   <2e-16 ***
## RushingYards -0.070949   0.007034  -10.09   <2e-16 ***
## ---
## Signif. codes:  0 '***' 0.001 '**' 0.01 '*' 0.05 '.' 0.1 ' ' 1
## 
## Residual standard error: 24.87 on 94 degrees of freedom
## Multiple R-squared:  0.5198, Adjusted R-squared:  0.5147 
## F-statistic: 101.8 on 1 and 94 DF,  p-value: < 2.2e-16
\end{verbatim}

The next model I reviewed was receiving yards. This model also had a low
Pr value, but the multiple R-squared was also significantly lower at
.3475. Both these models only looked at only one categorie when
predicting average draft position. I decided to keep running models
while adding categories to see if there was a better fit. During this
process I took into account the idea of overfiiting and tried to keep
the focus on only a few categories.

\begin{verbatim}
## 
## Call:
## lm(formula = ADP ~ ReceivingYards, data = rb)
## 
## Residuals:
##     Min      1Q  Median      3Q     Max 
## -72.174 -20.894   4.093  20.677  62.586 
## 
## Coefficients:
##                Estimate Std. Error t value Pr(>|t|)    
## (Intercept)    82.70620    4.42460  18.692  < 2e-16 ***
## ReceivingYards -0.11839    0.01673  -7.075 2.64e-10 ***
## ---
## Signif. codes:  0 '***' 0.001 '**' 0.01 '*' 0.05 '.' 0.1 ' ' 1
## 
## Residual standard error: 28.99 on 94 degrees of freedom
## Multiple R-squared:  0.3475, Adjusted R-squared:  0.3405 
## F-statistic: 50.05 on 1 and 94 DF,  p-value: 2.642e-10
\end{verbatim}

After running serveral other regression models, I finally ended up with
rushing yards and receiving yards together to predict average draft
position.

\begin{verbatim}
## 
## Call:
## lm(formula = ADP ~ RushingYards + ReceivingYards, data = rb)
## 
## Residuals:
##     Min      1Q  Median      3Q     Max 
## -88.339 -16.292   1.744  17.299  47.157 
## 
## Coefficients:
##                 Estimate Std. Error t value Pr(>|t|)    
## (Intercept)    96.259226   3.995527  24.092  < 2e-16 ***
## RushingYards   -0.055906   0.007629  -7.328 8.36e-11 ***
## ReceivingYards -0.060221   0.015570  -3.868 0.000204 ***
## ---
## Signif. codes:  0 '***' 0.001 '**' 0.01 '*' 0.05 '.' 0.1 ' ' 1
## 
## Residual standard error: 23.21 on 93 degrees of freedom
## Multiple R-squared:  0.5863, Adjusted R-squared:  0.5774 
## F-statistic: 65.91 on 2 and 93 DF,  p-value: < 2.2e-16
\end{verbatim}

As you can see from the below anova review, the model with rushing yards
and receiving yards together turned out to be the best fit. There was
the biggest reduction in the residual sum of squares (RSS) between the
first and second model. The Pr value was also very low and the most
significantly signficant when compared to all four models.

\begin{verbatim}
## Analysis of Variance Table
## 
## Model 1: ADP ~ RushingYards
## Model 2: ADP ~ RushingYards + ReceivingYards
## Model 3: ADP ~ RushingYards + ReceivingYards + RushingAttempts
## Model 4: ADP ~ RushingYards + ReceivingYards + RushingAttempts + ReceivingTargets
##   Res.Df   RSS Df Sum of Sq       F    Pr(>F)    
## 1     94 58139                                   
## 2     93 50083  1    8056.2 14.7123 0.0002306 ***
## 3     92 49860  1     222.9  0.4071 0.5250301    
## 4     91 49830  1      29.8  0.0545 0.8159674    
## ---
## Signif. codes:  0 '***' 0.001 '**' 0.01 '*' 0.05 '.' 0.1 ' ' 1
\end{verbatim}

\subsection{Ideas for Further
Research}\label{ideas-for-further-research}

There is a lot of room to expand the scope of this project in further
research. This project only analyzed one of the many positions in
football. I think it would be a great idea to provide the same type of
analysis for each position and then tie it all together to get the full
picture of a complete fantasy team. Each position could be compared to
each other to see which one is the most valuable and which positions to
prioritze when drafting. With these analytics, you could help prove the
common idea that drafting a kicker in the third round is a terrible
decision.

\subsection{Conclusion}\label{conclusion}

While there will always be several factors to decide on in real time
during the fantasy draft, it would be a good idea to develop a solid
game plan of the top running backs you believe have the best chance to
succeed before the acual draft. While making this list of running backs,
there should be a strong focus on his rushing and receiving yards from
the previous season. These categories from the previous year could be a
strong indictor of future success and should be factored into your
ranking. The model with rushing and receiving yards could also be
applied on future seasons. Hopefully with this knowledge, fantsy
football league memebers will be better prepared for their future
drafts.


\end{document}
